\documentclass{report}
\usepackage[a4paper,margin=1in]{geometry}
\usepackage{fancyhdr}
\usepackage{hyperref}
\usepackage{graphicx}
\usepackage{lipsum}
\usepackage{geometry}
\usepackage{enumitem}
\usepackage{colortbl}
\usepackage{xcolor}
\usepackage[french]{babel}

\geometry{a4paper, margin=1in}

\pagestyle{fancy}
\fancyhf{}
\lhead{IUT Lannion - SAÉ 3 Tchatator - Répartition des tâches}
\rhead{The void - 2024-25}

\title{\Huge Tchatator \\ {\LARGE Répartition des tâches}}

\begin{document}

\begin{titlepage}
    \begin{center}
        {\Huge\bfseries Tchatator} \\[1cm]
        {\Large\textit{Répartition des tâches}} \\[2cm]

        \includegraphics[width=0.4\textwidth]{../../../html/img/logo.png} \\[2cm]

        {\Large \textbf{A2.1 The void}} \\ [.2cm]
        {\large Mattéo KERVADEC } \\[.2cm]
        {\large Kylian Houedec } \\[.2cm]
        {\large Gabriel FROC } \\[.2cm]
        {\large Ewen JAIN } \\[.2cm]
        {\large Antoine GUILLERM } \\[.2cm]
        {\large Benjamin GERARD } \\[1cm]

        \textbf{IUT Lannion - SAÉ 3 2024-2025} \\[1.2cm]

        \textit{
        Le projet TripEnArvor a pour objectif de mettre en place un mécanisme d’échange de messages au format texte brut entre les professionnels et les visiteurs/clients de la plateforme. Ce mécanisme fonctionnera en mode asynchrone, permettant à chaque correspondant d’envoyer et recevoir des messages de manière décalée, sans nécessiter de connexion simultanée. 
		} \\[0.5cm]
		
		\textit{
        Ce document a pour but de présenter l'implication de travail de chacun sur le tchatator
		}

        \vfill  % Remplit l'espace vertical jusqu'en bas

        {\small Dernière modification : \today}
    \end{center}
\end{titlepage}

\chapter{Conception du protocole}

\begin{table}[ht]
\hspace{-1cm}  % Décale le tableau vers la gauche
\begin{tabular}{|p{3cm}|p{2cm}|p{2cm}|p{2cm}|p{2cm}|p{2cm}|p{2cm}|}
\hline
\textbf{Fonctionnalité} & \textbf{Mattéo Kervadec} & \textbf{Antoine Guillerm} & \textbf{Gabriel Froc} & \textbf{Ewen Jain} & \textbf{Kylian Houedec} & \textbf{Benjamin Gerard} \\
\hline
Modularité & 100\% & & & & & \\
\hline
Envoie de la réponse à l'utilisateur & 100\% & & & & & \\
\hline
Mise en place du socket & 100\% & & & & & \\
\hline
Gestion des options & 100\% & & & & & \\
\hline
Gestion des logs & 100\% & & & & & \\
\hline
Mise en place BDD & & & 100\% & & & \\
\hline
Authentification & & & & 100\% & & \\
\hline
Envoi de message & 100\% & & & & & \\
\hline
Réception de message non lu & & & & & & \\
\hline
Réception d'un historique & & & 100\% & & & \\
\hline
Modification d'un message & & & & & & \\
\hline
suppression d'un message & & & & 100\% & & \\
\hline
Blocage / Bannissement & & & & & & \\
\hline
Déblocage / Débannissement &  & & & & & \\
\hline
Limitation de requêtes & 100\% & & & & & \\
\hline
\end{tabular}
\caption{Répartition des tâches - conception}
\end{table}

\chapter{Codage et tests}

\begin{table}[ht]
\hspace{-1cm}  % Décale le tableau vers la gauche
\begin{tabular}{|p{3cm}|p{2cm}|p{2cm}|p{2cm}|p{2cm}|p{2cm}|p{2cm}|}
\hline
\textbf{Fonctionnalité} & \textbf{Mattéo Kervadec} & \textbf{Antoine Guillerm} & \textbf{Gabriel Froc} & \textbf{Ewen Jain} & \textbf{Kylian Houedec} & \textbf{Benjamin Gerard} \\
\hline
Mise en place du socket & 100\% & & & & & \\
\hline
Envoie de la réponse à l'utilisateur & 100\% & & & & & \\
\hline
Mise en place sur le serveur & & & & & 100\% & \\
\hline
Gestion des options & 100\% & & & & & \\
\hline
Gestion des logs & 100\% & & & & & \\
\hline
Mise en place BDD & & & 100\% & & & \\
\hline
Authentification & 35\% & & & 65\% & & \\
\hline
Envoi de message & 100\% & & & & & \\
\hline
Réception de message non lu & & & & & & \\
\hline
Réception d'un historique & & & 100\% & & & \\
\hline
Modification d'un message & & & & & & \\
\hline
suppression d'un message & & & & 100\% & & \\
\hline
Blocage / Bannissement & & & & & & \\
\hline
Déblocage / Débannissement &  & & & & & \\
\hline
Limitation de requêtes & 100\% & & & & & \\
\hline
\end{tabular}
\caption{Répartition des tâches - codages et tests}
\end{table}

\chapter{Documentation}

\begin{table}[ht]
\hspace{-1cm}  % Décale le tableau vers la gauche
\begin{tabular}{|p{3cm}|p{2cm}|p{2cm}|p{2cm}|p{2cm}|p{2cm}|p{2cm}|}
\hline
\textbf{Fonctionnalité} & \textbf{Mattéo Kervadec} & \textbf{Antoine Guillerm} & \textbf{Gabriel Froc} & \textbf{Ewen Jain} & \textbf{Kylian Houedec} & \textbf{Benjamin Gerard} \\
\hline
Documentation des fonctions & 65\% & & 15\% & 15\% & & \\
\hline
Documentation utilisateur & 50\% & 50\% & & & & \\
\hline
Documentation technique & 100\% & & & & & \\
\hline
Documentation des protocoles & & & & & & \\
\hline
\end{tabular}
\caption{Répartition des tâches - documentation}
\end{table}

\end{document}