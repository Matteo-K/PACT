\documentclass{report}
\usepackage[a4paper,margin=1in]{geometry}
\usepackage{fancyhdr}
\usepackage{hyperref}
\usepackage{graphicx}
\usepackage{lipsum}
\usepackage{geometry}
\usepackage{enumitem}
\usepackage[french]{babel}

\geometry{a4paper, margin=1in}

\pagestyle{fancy}
\fancyhf{}
\lhead{IUT Lannion - SAÉ 3 Tchatator - documentation utilisateur}
\rhead{The void - 2024-25}

\title{\Huge Tchatator \\ {\LARGE Documentation Technique}}

\begin{document}

\begin{titlepage}
    \begin{center}
        {\Huge\bfseries Tchatator} \\[1cm]
        {\Large\textit{Documentation utilisateur}} \\[2cm]

        \includegraphics[width=0.4\textwidth]{../../../html/img/logo.png} \\[2cm]

        {\Large \textbf{A2.1 The void}} \\ [.2cm]
        {\large Mattéo KERVADEC } \\[.2cm]
        {\large Kylian Houedec } \\[.2cm]
        {\large Gabriel FROC } \\[.2cm]
        {\large Ewen JAIN } \\[.2cm]
        {\large Antoine GUILLERM } \\[.2cm]
        {\large Benjamin GERARD } \\[1cm]

        \textbf{IUT Lannion - SAÉ 3 2024-2025} \\[1.2cm]

        \textit{
        Le projet TripEnArvor a pour objectif de mettre en place un mécanisme d’échange de messages au format texte brut entre les professionnels et les visiteurs/clients de la plateforme. Ce mécanisme fonctionnera en mode asynchrone, permettant à chaque correspondant d’envoyer et recevoir des messages de manière décalée, sans nécessiter de connexion simultanée. 
		} \\[0.5cm]
		
		\textit{
        Ce document fournit une explication détaillée sur l'installation, l'utilisation et la configuration du service.
		}

        \vfill  % Remplit l'espace vertical jusqu'en bas

        {\small Dernière modification : \today}
    \end{center}
\end{titlepage}

\tableofcontents
\newpage

\chapter{Introduction}

Tchatator est un service de discussion asynchrone en C utilisant les sockets et une API permettant aux clients, professionnels et administrateurs d'envoyer et de recevoir des messages. Ce document fournit une explication détaillée sur l'installation, l'utilisation et la configuration du service.

\section{Installation}

\subsection{Prérequis}

Avant d'installer le service, assurez-vous d'avoir les prérequis nécessaires installés sur votre machine. Cela inclut la bibliothèque json-c, qui est utilisée pour le traitement des données JSON, ainsi que d'autres outils nécessaires à la compilation et à l'exécution des scripts.

\paragraph{Prérequis communs :}
\begin{itemize}

	\item \textbf{Système d'exploitation :}  Linux
	\item \textbf{Compilateur :}  gcc pour compiler le code C
	\item \textbf{Bibliothèque json-c:} Bibliothèque pour le traitement des données JSON.

\end{itemize}

\paragraph{Pour installer la bibliothèque json-c :}

Sur une distribution Linux (comme Ubuntu/Debian), vous pouvez installer json-c avec apt :

\begin{verbatim}
sudo apt-get update
sudo apt-get install libjson-c-dev
\end{verbatim}

\subsection{Installation du Service côté Serveur}

Pour installer et exécuter le serveur, suivez les étapes suivantes.

\paragraph{a) Télécharger les fichiers du projet :}

Clonez ou téléchargez le dépôt contenant le code source du projet.

\begin{verbatim}
git clone https://github.com/Matteo-K/PACT.git
cd ./PACT/tchatator
\end{verbatim}

\paragraph{b) Compilation et exécution du serveur :} Le serveur utilise le script tchatator.sh pour compiler le programme. Ce script gère la configuration, l'inclusion des bibliothèques nécessaires (comme json-c), et la compilation du code source.

\begin{enumerate}
	\item Ouvrez un terminal et naviguez jusqu'au répertoire PACT/tchatator.
	\item Exécutez le script tchatator.sh pour compiler le programme.
\end{enumerate}

\begin{verbatim}
chmod +x tchatator.sh
./tchatator.sh
\end{verbatim}

Le script s'assurera que toutes les dépendances nécessaires sont présentes, comme la bibliothèque json-c, et compilera le code C dans un exécutable. Ainsi ce dernier sera exécuter.

\paragraph{c) Options d'exécution}

\begin{verbatim}
	./tchatator.sh [option]
\end{verbatim}

\large \textbf{OPTIONS}

\begin{itemize}[label={}]
	\item \textbf{-h, --help}
	\begin{itemize}
		\item afficher l’aide
	\end{itemize}
	\item \textbf{-v, --version}
	\begin{itemize}
		\item afficher la version actuelle du tchatator
	\end{itemize}
	\item \textbf{-b, --verbose}
	\begin{itemize}
		\item afficher les logs
	\end{itemize}
\end{itemize}

\paragraph{d) Service côté Client :}

Le script user.sh est utilisé pour lancer un terminal interactif pour l'utilisateur. Ce script gère l'installation de telnet, l'ouverture d'une session utilisateur et la connexion au serveur de chat. Il suffit de rendre le script exécutable et de l'exécuter.

\begin{enumerate}
	\item Ouvrez un terminal et naviguez jusqu'au répertoire PACT/tchatator.
	
	\item Rendez le script user.sh exécutable :
	\begin{verbatim}
		chmod +x user.sh
	\end{verbatim}
	
	\item Exécutez le script pour ouvrir un terminal et commencer à interagir avec le serveur.
	
	\begin{verbatim}
		./user.sh
	\end{verbatim}
\end{enumerate}

\subsection{Dépannage}

Voici quelques problèmes courants et leurs solutions possibles :

\begin{itemize}
	\item Problème de dépendances (json-c) :
	\begin{itemize}
		\item Assurez-vous que la bibliothèque json-c est correctement installée.
		\item Si vous avez des problèmes lors de la compilation, vérifiez que le répertoire de l'en-tête json-c est correctement inclus dans les chemins du compilateur.
	\end{itemize}
	\item Problème de connexion au serveur :
	\begin{itemize}
		\item Vérifiez si le serveur est en cours d'exécution et écoute bien sur le port attendu.
	\end{itemize}
\end{itemize}

\chapter{Démarrage}

\section{Exécution du service avec Telnet}

Telnet est un protocole de communication réseau qui permet à un utilisateur de se connecter à un ordinateur distant via un terminal en ligne de commande. Il offre la possibilité d’exécuter des commandes sur le serveur distant.

\begin{verbatim}
	telnet the-void.ventsdouest.dev 8081
\end{verbatim}

\subsection{Connexion au service}

Le principe d'identification dans Tchatator permet de sécuriser les échanges en assurant que seuls les utilisateurs autorisés peuvent participer aux conversations. Il permet aussi de personnaliser l'expérience, gérer les rôles (clients, professionnels, administrateurs) et garantir la traçabilité des interactions.

\begin{verbatim}
	LOGIN: <clé_api>
\end{verbatim}

\begin{itemize}
	\item clé api : clé api de l'utilisateur voulant se connecter
\end{itemize}

La connexion au service renvoie un token si tout se passe bien.

Exemple:
\begin{verbatim}
{
  "statut" : "200/OK",
  "tokken" : "Age8Ku"
}
\end{verbatim}

\paragraph{Réponses possibles après la tentatives de connexion}

\begin{itemize}
	\item 200/OK : accès autorisé
	\item 400/TOO \_ MANY \_ ARGS : Trop de paramètres fournit
	\item 400/MISSING \_ ARGS : Pas assez de paramètres fournit
	\item 403/CLIENT \_ BANNED : accès refusé, client banni
	\item 403/CLIENT \_ BLOCKED : accès refusé, client bloqué
	\item 429/QUOTA \_ EXCEEDED : Quota dépassé pour la clé API
\end{itemize}

\subsection{Déconnexion du service}

La déconnexion du service permet de sécuriser les sessions utilisateur en mettant fin à l'accès à leurs données et fonctionnalités après une période d'inactivité ou lorsque l'utilisateur choisit de se déconnecter. Cela protège les informations personnelles et prévient les accès non autorisés.

\begin{verbatim}
	BYE BYE: <tokken>
\end{verbatim}

\begin{itemize}
	\item tokken : identifiant de l'utilisateur, reçu lors de la connexion
\end{itemize}

Exemple:
\begin{verbatim}
{
  "statut" : "200/OK"
}
\end{verbatim}

\paragraph{Réponses possibles après la tentatives de déconnexion}

\begin{itemize}
	\item 200/OK : accès autorisé
	\item 400/TOO \_ MANY \_ ARGS : Trop de paramètres fournit
	\item 400/MISSING \_ ARGS : Pas assez de paramètres fournit
	\item 403/CLIENT \_ BANNED : accès refusé, client banni
	\item 403/CLIENT \_ BLOCKED : accès refusé, client bloqué
\end{itemize}

\subsection{Avoir de l’aide}

La commande AIDE permet d'afficher de l'aide et des informations sur le fonctionnement du service ou de l'application. Elle fournit des instructions sur l'utilisation des différentes fonctionnalités et peut guider l'utilisateur en cas de besoin, facilitant ainsi la compréhension et la gestion du service.

\begin{verbatim}
	AIDE: <fonctionnalité>
\end{verbatim}

\begin{itemize}
	\item fonctionnalité : une des fonctionnalités du tchatator
\end{itemize}

En cours de développement

\subsection{Envoyer un message}

Envoyer un message entre un professionnel et un membre (ou inversement) permet de faciliter la communication bidirectionnelle. Cela permet aux professionnels de répondre aux questions, fournir des informations ou clarifications, tandis que les membres peuvent poser des questions ou exprimer des besoins spécifiques. Cette interaction renforce l'efficacité du service et améliore l'expérience utilisateur.

\begin{verbatim}
	MSG: <tokken> | <clé api destiantaire> | <message>
\end{verbatim}

\begin{itemize}
	\item tokken : identifiant de l'utilisateur après une connexion
	\item clé api destiantaire : identifiant du destinataire qui n'est pas forcément connecté.
	\item message : message émit pour le destinataire
\end{itemize}

\paragraph{Réponses possibles après la tentatives d'envoi de message}

\begin{itemize}
	\item 200/OK : accès autorisé
	\item 400/TOO \_ MANY \_ ARGS : Trop de paramètres fournit
	\item 400/MISSING \_ ARGS : Pas assez de paramètres fournit
	\item 401/UNAUTH/UNKNOWN \_ RECIPIENT : Destinataire inconnue
	\item 403/CLIENT \_ BANNED : accès refusé, client banni
	\item 403/CLIENT \_ BLOCKED : accès refusé, client bloqué
	\item 403/UNAUTHORIZED \_ USE : utilisateur non autorisé d'accéder à cette REQUÊTE.
	\item 416/MISFMT : Message mal formaté ou trop long
	\item 429/QUOTA \_ EXCEEDED : Quota dépassé pour la clé API
	\item 500/INTERNAL\_ SERVER \_ ERROR : Erreur du côté du serveur
\end{itemize}

\subsection{Consulter l’historique de message}

Consulter l’historique des messages permet de garder une trace des échanges passés, ce qui facilite le suivi des conversations, la résolution de problèmes et la référence à des informations précédemment partagées. Cela améliore la continuité du service et aide à éviter les malentendus ou répétitions dans les interactions.

\begin{verbatim}
	HISTORIQUE: <tokken>
\end{verbatim}

\begin{itemize}
	\item tokken : identifiant de l'utilisateur après une connexion
\end{itemize}

\subsection{Arrêt du serveur}

La commande STOP permet à l'administrateur d'arrêter le serveur de manière contrôlée. Cela garantit que toutes les connexions et les processus en cours sont terminés proprement, évitant ainsi la perte de données et assurant la sécurité du système avant son arrêt complet.

\begin{verbatim}
	STOP:
\end{verbatim}

En cours de développement (sécurisation)

\end{document}